\chapter{Evaluation}

In this chapter we give a final evaluation regarding the results achieved during the course of this work. Initially, we provide an approximate test coverage achieved on the Dynamics 365 for Finance and Operation application with the analyzed business processes. Then, we elaborate on the total amount of resources needed (in terms of time) in order to acquire the required knowledge necessary to work and develop on the application and to apply said knowledge in order to create the required test cases. Finally, we give a insight regarding the usefulness of the project as a whole with a particular focus on the applicability that it could have in a project with a tighter schedule.

%********************************** %First Section  **************************************
\section{Coverage} 

Providing an accurate approximation on the total coverage of our test suite is a particularly complex task given the large and ever-expanding nature of the application that we worked on. In order to discuss this task as precisely as possible we will evaluate this topic in relation to three different metrics: total coverage of the chosen business processes (under normal conditions), amount of covered edge cases and degree of coverage regarding the application as a whole. The five business processes that we have chosen as a starting point for our work were covered completely in relation to the initially defined requirements. Because of time constraint, we left out some additional activities...



%********************************** %Second Section  **************************************
\section{Resources} 

When we analyze the (temporal) resources needed to complete this project it is important to take into consideration some concepts that influenced the final outcome of the work. In this regard we want to differentiate two aspects that play a role on this subject: time to acquire the knowledge and document, time to test.
The first one relates to the resources necessary for a new team member to acquire the knowledge required in order to apply the testing techniques on the Dynamics 365 for Finance and Operation application and to create the documentation to describe the process. Insight regarding the acquisition of this knowledge is particularly useful for the reader to have a more precise outlook regarding the scope of the work and for a company to decide if it wants to start implementing this kind of techniques on their applications. Regarding this first concept it is important to notice that a big portion of the project time was spent familiarizing with the application, completing the necessary training activities and creating the documentation needed to share the final know-how (indicatively around 60\% of the total time). This set of tasks was naturally of central importance for us in order to develop effectively and, in the end, to provide some useful knowledge to the company. However it is easy to understand how a team member already part of the company would not need all the training activities that were previously described. Furthermore, the process of creating the documentation that describes the steps for the creation of a test case and the particulars of the used framework has to be performed only once and can later be uses indefinitely by the other members.
The second aspect that we want to describe is related to the time required for a developer (already in possession of the previously mentioned know-how) to implement the described testing activities for a new business process (which very important for the company in order to have a general idea of the expected amount of work needed when a new testing activity is started). We have estimated that a testing activity (from the creation of the first recording to the end of the expansion process) can be concluded in a matter of hours if there is a good underlying knowledge of the underlying business process and on what has to be tested and expanded. Giving a more precise estimation if fairly difficult given the fact that the size (and variability) of the tested feature(s) strongly influences the total time required to complete the testing activities. Naturally, our development process (around 40\% of the total work) was much longer because of the learning curve involved (it was a first time approach, no previous know-how was provided) and the number of different business processes considered. 

%********************************** %Third Section  **************************************
\section{Usefulness and Applicability} 

The project as a whole has certainly provided useful information to the company regarding the testing activities that could potentially be implemented in their future and current projects. The fact that we have provided a well documented overlook on the whole process is naturally a positive aspect particularly helpful in relation to the scarce availability of official documentation regarding certain technical aspects of the application. 
In regard to the general applicability of this work we should take into consideration the fact that we worked on a project developed internally by the company and not strictly influenced from deadlines imposed by a client (as mentioned before, Trade+ is developed as a starting point for customers that can request their own personalization to be built on top of it). Applying testing techniques in projects that have tighter schedules that need to be followed in order to deliver specific functionalities in time would certainly be a more difficult task and would require careful planning by the company. We will further expand this particular limitation in the following chapter.