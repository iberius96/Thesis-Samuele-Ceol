\chapter*{Abstract} 

The size of modern software applications often requires particular focus to be placed on various testing techniques in order to ensure that high standards of quality are respected and to prevent the release of code containing possible defects or presenting unwanted behaviors. 
In the case of Enterprise Resource Planning (ERP) the employment of said practices is of utmost importance given the dramatic impact the failure of such a system can have on a company. 
ERP applications represent the heart of many modern businesses that use this kind of tools to complete a wide array of core activities necessary for the company success and survival. Examples of such activities are: 
Finance and accounting, production, distribution, sales, business intelligence, customer services, human resources management. 
It is easy to understand how an issue, even in a single one of those modules, could result in a huge loss of a company time and internal resources. 
For a team of developers successfully implementing testing related techniques often means finding a balance between having a fair amount of test coverage and respecting the time constraint imposed by the client needs for a specific project. We work on a vertical solution (called Trade+), developed by Würth Phoenix and based on Microsoft Dynamics 365 for Finance and Operations. The intent of the project is to cover a sample of defined core processes with a series of test cases aimed at ensuring the expected operativity of some aspects of the application in range of different scenarios. We start by using tools provided within the application to translate high-level interactions with the system into reusable task recordings. This interactions represent the analyzed business process and form the basis for the developed tests. We then convert these recordings into coded test cases that are expanded with tailored code specifically created to verify the behavior of the system when exposed to an array of different condition and edge cases. Finally, we integrate our test cases with the nightly builds in order to apply regression testing techniques on the subset of functionalities that we have covered. 

\selectlanguage{ngerman}
\chapter*{Zusammenfassung}

Die Größe moderner Softwareanwendungen erfordert oft ein besonderes Augenmerk auf verschiedene Testtechniken, um sicherzustellen, dass hohe Qualitätsstandards eingehalten werden und um die Freigabe eines Codes mit möglichen Fehlern oder unerwünschtem Verhalten zu verhindern. 
Im Falle von Enterprise Resource Planning (ERP) ist der Einsatz dieser Praktiken von größter Bedeutung, da der Ausfall eines solchen Systems dramatische Auswirkungen auf ein Unternehmen haben kann. 
ERP-Anwendungen stellen das Herzstück vieler moderner Unternehmen dar, die diese Art von Tools einsetzen, um eine breite Palette von Kernaktivitäten abzuschließen, die für den Erfolg und das Überleben des Unternehmens notwendig sind. Beispiele für solche Aktivitäten sind: 
Finanz- und Rechnungswesen, Produktion, Distribution, Vertrieb, Business Intelligence, Kundenservice, Personalmanagement. 
Es ist leicht zu verstehen, wie ein Problem, selbst in einem einzelnen dieser Module, zu einem enormen Verlust von Zeit und internen Ressourcen eines Unternehmens führen kann. 
Für ein Team von Entwicklern gilt es, für die Implementierung von testbezogenen Techniken in einem bestimmten Projekt, ein Gleichgewicht zwischen einer guten Testabdeckung und der Einhaltung der vorgegebenen Zeitvorgaben des Kunden zu finden.
Wir arbeiten an einer vertikalen Lösung (Trade+), die von Würth Phoenix entwickelt wurde und auf "Microsoft Dynamics 365 for Finance and Operations" basiert. Ziel des Projekts ist es, eine Stichprobe definierter Kernprozesse mit einer Reihe von Testfällen abzudecken, um die erwartete Funktionsfähigkeit einiger Aspekte der Anwendung in verschiedenen Szenarien sicherzustellen. 
Zunächst verwenden wir bereitgestellten Tools, um hochrangige Interaktionen mit dem System in wiederverwendbare "Task Recordings" zu übersetzen. Diese Interaktionen stellen den analysierten Geschäftsprozess dar und bilden die Grundlage der entwickelten Tests.
Diese Aufzeichnungen werden in codierte Testfälle umgewandelt und mit einem dafür spezifisch entwickeltem Code erweitert werden, um das Verhalten des Systems zu überprüfen wenn es einer Reihe von verschiedenen Zustands- und Kantenfällen ausgesetzt ist. 
Schließlich integrieren wir unsere Testfälle mit den nächtlichen "Builds" (nightlies), um Regressionstests auf die Teilmenge der Funktionalitäten anzuwenden, die abgedeckt wurden. 

\selectlanguage{italian}
\chapter*{Sommario}

Le dimensioni delle moderne applicazioni software portano spesso a dover porre particolare enfasi riguardo le varie tecniche di testing al fine di garantire il rispetto di elevati standard di qualità e di evitare il rilascio di codice contenente possibili difetti o comportamenti indesiderati. 
Nel caso dell'Enterprise Resource Planning (ERP) l'impiego di tali pratiche è della massima importanza dato il drammatico impatto che il fallimento di un'tale sistema può avere su un'azienda. 
Le applicazioni ERP rappresentano il cuore di molti business moderni che utilizzano questo tipo di programmi per completare una vasta gamma di attività fondamentali per il loro successo e la loro sopravvivenza. 
Esempi di tali attività sono: Finanza e contabilità, produzione, distribuzione, vendite, business intelligence, servizio clienti, gestione delle risorse umane. 
Risulta quindi semplice comprendere come un problema, anche in un singolo di questi moduli, possa comportare un'enorme perdita di tempo e di risorse per l'azienda interessata. 
Per un team di sviluppatori implementare con successo le varie tecniche di testing significa spesso trovare un equilibrio tra l'ottenimento di una discreta copertura di test e il rispetto dei vincoli temporali imposti dalle esigenze del cliente per uno specifico progetto. 
Lavoriamo su una soluzione verticale (denominata Trade+), sviluppata da Würth Phoenix e basata su Microsoft Dynamics 365 for Finance and Operations. L'intento del progetto è quello di coprire un campione di processi "core" con una serie di test case finalizzati a garantire l'operatività di alcuni aspetti dell'applicazione in diversi scenari. 
Iniziamo utilizzando gli strumenti forniti all'interno dell'applicazione per tradurre interazioni di alto livello con il sistema in task recordings riutilizzabili. 
Queste interazioni rappresentano i processi di business analizzati e costituiscono la base per dei test case sviluppati. 
Le registrazioni vengono quindi convertite in test codificati che sono poi espansi con codice personalizzato creato appositamente per verificare il comportamento del sistema quando viene esposto a una serie di condizioni e casi limite specifici. 
Infine, integriamo i nostri test case con le build notturne (nightlies) al fine di applicare tecniche di regression testing sul sottoinsieme di funzionalità che abbiamo coperto. 

\selectlanguage{english}