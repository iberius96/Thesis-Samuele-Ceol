\chapter*{Abstract} 

The size of modern software applications often requires particular focus to be placed on various testing techniques in order to ensure that high standards of quality are respected and to prevent the release of code containing possible defects or presenting unwanted behaviors. 
In the case of Enterprise Resource Planning (ERP) the employment of said practices is of utmost importance given the dramatic impact the failure of such a system can have on a company. 
ERP applications represent the heart of many modern businesses that use this kind of tools to complete a wide array of core activities necessary for the company success and survival. Examples of such activities are: 
Finance and accounting, production, distribution, sales, business intelligence, customer services, human resources management. 
It is easy to understand how an issue, even in a single one of those modules, could result in a huge loss of a company time and internal resources. 
For a team of developers successfully implementing testing related techniques often means finding a balance between having a fair amount of test coverage and respecting the time constraint imposed by the client needs for a specific project. We work on a vertical solution (called Trade+), developed by Würth Phoenix and based on Microsoft Dynamics 365 for Finance and Operations. The intent of the project is to cover a sample of defined core processes with a series of test cases aimed at ensuring the expected operativity of some aspects of the application in range of different scenarios. We start by using tools provided within the application to translate high-level interactions with the system into reusable task recordings. This interactions represent the analyzed business process and form the basis for the developed tests. We then convert these recordings into coded test cases that are expanded with tailored code specifically created to verify the behavior of the system when exposed to an array of different condition and edge cases. Finally, we integrate our test cases with the nightly builds in order to apply regression testing techniques on the subset of functionalities that we have covered. 

\selectlanguage{ngerman}
\chapter*{Zusammenfassung}

Lorem ipsum dolor sit amet, consectetur adipiscing elit. Phasellus scelerisque turpis a posuere pulvinar. Sed nec sodales justo. Pellentesque habitant morbi tristique senectus et netus et malesuada fames ac turpis egestas. Nullam tempus lectus justo, et vehicula mauris semper nec. Nam lorem leo, vehicula ut arcu non, pharetra varius enim. Donec mollis consequat faucibus. Integer aliquet erat egestas mauris hendrerit pretium.

Donec ullamcorper feugiat lacus. Interdum et malesuada fames ac ante ipsum primis in faucibus. Suspendisse blandit ut enim sit amet scelerisque. Aenean scelerisque vulputate dolor. Nulla nibh magna, facilisis ut mauris quis, volutpat tristique augue. Etiam ornare nec nisi ut bibendum. Sed sollicitudin sapien neque, ut suscipit felis lacinia in. Etiam congue augue eu ante tempus, ut elementum nunc posuere. Maecenas feugiat iaculis enim et porttitor.

\selectlanguage{italian}
\chapter*{Sommario}

Le dimensioni delle moderne applicazioni software richiedono spesso particolare attenzione riguardo le varie tecniche di testing al fine di garantire il rispetto di elevati standard di qualità e di evitare il rilascio di codice contenente possibili difetti o comportamenti indesiderati. 
Nel caso dell'Enterprise Resource Planning (ERP) l'impiego di tali pratiche è della massima importanza dato il drammatico impatto che il fallimento di un tale sistema può avere su un'azienda. 
Le applicazioni ERP rappresentano il cuore di molti business moderni che utilizzano questo tipo di strumenti per completare una vasta gamma di attività fondamentali per il proprio successo e la propria sopravvivenza. Esempi di tali attività sono: 
Finanza e contabilità, produzione, distribuzione, vendite, business intelligence, servizio clienti, gestione delle risorse umane. 
Risulta quindi semplice comprendere come un'problema, anche in un singolo di questi moduli, possa comportare un'enorme perdita di tempo e di risorse  per l'azienda. 
Per un team di sviluppatori implementare con successo le relative tecniche di testing significa spesso trovare un equilibrio tra l'avere un discreto test coverage e rispettare i vincoli temporali imposti dalle esigenze del cliente per uno specifico progetto. 
Lavoriamo su una soluzione verticale (denominata Trade+), sviluppata da Würth Phoenix e basata su Microsoft Dynamics 365 for Finance and Operations. L'intento del progetto è quello di coprire un campione di core processes definiti con una serie di test case finalizzati a garantire l'operatività attesa di alcuni aspetti dell'applicazione in diversi scenari. Iniziamo utilizzando gli strumenti forniti all'interno dell'applicazione per tradurre le interazioni di alto livello con il sistema in registrazioni di attività riutilizzabili. Queste interazioni rappresentano il processo di business analizzato e costituiscono la base per i test sviluppati. Queste registrazioni vengono poi convertite in casi di test codificati che vengono espansi con codice personalizzato creato appositamente per verificare il comportamento del sistema quando viene esposto a una serie di condizioni e casi limite diversi. Infine, integriamo i nostri casi di test con le costruzioni notturne al fine di applicare tecniche di test di regressione sul sottoinsieme di funzionalità che abbiamo coperto. 

\selectlanguage{english}