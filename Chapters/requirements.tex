\chapter{Project Requirements}

The result of the Diagnostic phase was the definition (through the Project Plan) of the various requirements that needed to be completed before the end of the internship period. This definition was further refined in the Analysis phase where a list of core processes of Trade+ together with their testing requirements was created. The test cases not only needed to verify that the tested features were able to function under a range of different scenarios but also that the expected error or warning message was returned in the case of an unforeseen interaction with the system. The (initial) minimum  requirement consisted in checking that the application could successfully complete all the tested processes under "normal" conditions. This meant creating the XML recordings, converting them into X++ code, adding code to verify that the appropriate changes were performed on the database at the end of each process and obtaining a successful outcome after the run each test. A second requirement consisted in testing for edge cases and unexpected behaviors and verifying the correct response of the system during this particular scenarios. The purpose of this activities was not only to check if Trade+ was able to handle "non-convectional" situations but also to verify that appropriate and informative messages were prompted to the end user in order to successfully complete a given activity. Each process had to be broken down into individual sub-activities, each of which needed to be studied in order to understand which kind of situations could lead to potential issues. The text cases were then expanded accordingly following the need of this second requirement. Finally, as a last testing related requirement, we verified to which degree the different processes were able to correctly interact with each other by linking them with one another and creating a single flow of actions that covered the purchase of goods for a specific location, the transfer of said items to another warehouse, the creation of a sales order to a specific customer and (at the end) the return of the goods to the selling warehouse.

As mentioned before, a secondary objective (part of the Development phase) was the integration of the developed test cases into the nightly builds. Said integration had to be started after the completion of the test case development and was divided into three stages: Acquisition of the technical knowledge needed to start the integration process, application of the acquired knowledge on a sample test to verify the correct functioning of all the component needed to complete this process and, as a final step, integration of all the previously developed test cases. 

Finally, as part of the Deployment phase, we created the documentation needed to pass the knowledge derived from this experience to the company through a series of final meetings during which this acquired know-how was presented.
