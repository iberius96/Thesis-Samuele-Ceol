\addtocontents{toc}{\protect\enlargethispage{\baselineskip}}

\chapter{Conclusion}

In this chapter we wrap up the previously described content by providing a short summary to recap the various activities that were carried out during the course of this project. 
In this work we described the approach that was taken in order to implement a set testing techniques in a Dynamics 365 for Finance and Operations environment (particularly in the context of the Trade+ solution). 
We started by providing an in-depth technical description of the used framework, application and lifecycle management tool. 
Then, we described the various training activities and initial meetings with the company that allowed us to define the structure and requirements of the project and to obtain the required knowledge in order to start the work on the application. 
After this initial description, we illustrated the process of creating the first task recordings in the browser client, saving them as XML files and converting them into coded test cases using the tool provided by the Visual Studio environment. 
This allowed us to perform and apply the first component/feature testing techniques by generating a set of executable test cases that worked in isolation (from the underlying database and from one another). 
Subsequently, we introduced the concept of the expansion of test cases to verify the reaction of the system under various edge scenarios. At the end of this activity we worked to allow the various test cases to interact with one another and function as a group that represented a single and coherent workflow covering all the interested business processes (integration testing).  
Finally, we integrated the set of developed test cases into the nightly builds to allow for the application of regression testing techniques. 
During the whole project we documented the various activities that were performed on a day-to-day basis and then we passed the relevant information regarding the acquired know-how to the company. 
The final results of this work showed us that the implementation of the initially defined requirements was not only possible in practice, but also a viable and useful approach to follow during the whole development process in order to increase the quality and reliability of the final product. 
Furthermore, it is important to notice that the application of the previously described techniques is not necessarily a binary approach, meaning that the company has the ability (particularly with the provided documentation) to implement this activities with various degree of intensity depending on the requirement of a given project and the particulars of the external deadlines imposed by the customer. 
Because of this reason we believe that this topic may be worth exploring for most companies producing ERP software that are interested in improving the overall quality of their final products.


